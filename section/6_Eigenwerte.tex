%%%%%%%%%%%%%%%%%%%%%%%%%%%%%%%%%%%%%%%%%%%%%%%%
% Eigenwerte und Eigenvektoren
%%%%%%%%%%%%%%%%%%%%%%%%%%%%%%%%%%%%%%%%%%%%%%%%
\section{Eigenwerte und Eigenvektoren}
	Wenn eine Abbildung auf denselben Punkt fällt ($\vec{v} = \vec{v}'$), nennt man dies Eigenfixpunkt.
	Der Eigenvektor $\vec{v}$ zeigt nun in diese Richtung (als Gerade) und der Eigenwert $\lambda$ gibt den Faktor an, mit der in
	diese Richtung gezeigt wird. D.h. es gilt
	\begin{equation*}
		Av = \lambda v
	\end{equation*}

	\textbf{Def.:} Eine Matrix heisst diagonalisierbar, wenn es eine Basis aus den Eigenvektoren gibt.\\

	\textbf{Def.:} Symmetrische Matrizen sind diagonalisierbar (die Eigenwerte stehen in der Diagonalen). 
	Es gibt eine orthonormierte Eigenvektorbasis.

	\subsection{Berechnen der Eigenwerte}
		\begin{enumerate}
			\item Determintante ausrechnen
			\item Gleichung lösen (Lösungen = Eigenwerte)
		\end{enumerate}
		\begin{equation*}
			det(A - \lambda E) = 0
		\end{equation*}
		
		Beispiel:
		\begin{equation*}
			A = \left(\begin{array}{cc}
				a_1 & b_1\\
				a_2 & b_2
			\end{array}\right) \\ \Longrightarrow\\
			\left|\begin{array}{cc}
				a_1 - \lambda & b_1\\
				a_2 & b_2 - \lambda
			\end{array}\right| = 0 \\ \Longrightarrow\\
			\lambda^2 -\lambda(a_1 + b_2) + a_1b_2 - a_2b_1 = 0
		\end{equation*}
		
		\textbf{Def.:} $det(A - \lambda E)$ heisst charakteristisches Polynom $\chi_A$\\

	\subsection{Berechnen der Eigenvektoren}
		\textbf{Für jeden Eigenwert $\lambda_i$}  Gleichungssystem aufstellen und mit Gauss auflösen $\Rightarrow$ eine Zeile
		verschwindet $\Rightarrow \infty$ Lösungen $\Rightarrow$ Wert von verschwundener Zeile frei wählbar.
		\begin{equation*}
			(A - \lambda_i E)\vec{v_i} = 0
		\end{equation*}

		Beispiel:
		\begin{equation*}
			\left|\begin{array}{ccc}
				v_1 - 2v_2 & = & 0\\
				0 & = & 0
			\end{array}\right| \\
			\Rightarrow \\
			v_2 \text{ ist frei wählbar} \\
			\Rightarrow \\
			\begin{array}{c}
				v_1 = 2v_2\\
				v_2 = x
			\end{array} \\
			\Rightarrow \\
			\vec{v} = \left(\begin{array}{c}
				v_1 \\
				v_2
			\end{array}\right) = \left(\begin{array}{c}
				2v_2\\
				v_2
			\end{array}\right) = \left(\begin{array}{c}
				2x \\
				x
			\end{array}\right) = x \cdot \left(\begin{array}{c}
				2 \\
				1
			\end{array}\right)
		\end{equation*}

	\subsection{Potenzrechen von Matrizen}
		\begin{equation*}
			A^k = T^{-1}{A'}^kT = T^{-1}\left(\begin{array}{ccc}
				{\lambda_1}^k & & 0\\
				& \ddots & \\
				0 & & {\lambda_n}^k
			\end{array}\right) T
		\end{equation*}
		\begin{equation*}
			A' = \left(\begin{array}{ccc}
				{\lambda_1}^k & & 0\\
				& \ddots & \\
				0 & & {\lambda_n}^k
			\end{array}\right) 
			A' \text{ in Eigenvektoren-Basis}
		\end{equation*}
		\begin{equation*}
			T = (Ev_1, Ev_2, \ldots, Ev_n)^{-1}
		\end{equation*}

	\subsection{Rekursionsformel}
		$x_{n+3} = 4x_n - 11x_{n+1} + 6x_{n+2}$\\
		\begin{enumerate}
			\item Matrix/Vektor Schreibweise : $\left(\begin{array}{c}
					x_{n+3}\\
					x_{n+2}\\
					x_{n+1}
				\end{array}\right) = A\left(\begin{array}{c}
					x_{n+2}\\
					x_{n+1}\\
					x_n
				\end{array}\right) \qquad \qquad A = \left(\begin{array}{ccc}
					6 & -11 & 4\\
					1 & 0 & 0\\
					0 & 1 & 0
				\end{array}\right)$

			\item $\vec{u}$ ein Eigenvektor von $A$ \qquad $A^nu = \lambda^nu$ \\
				$\left(\begin{array}{c}
					x_{n+2}\\
					x_{n+1}\\
					x_n
				\end{array}\right) = 
				a_1{\lambda_1}^n\vec{u_1}+a_2{\lambda_2}^n\vec{u_2}+ a_3{\lambda_3}^n\vec{u_3} =
				a_1{\lambda_1}^n\left(\begin{array}{c}
					u_{11}\\
					u_{12}\\
					u_{13}
				\end{array}\right) + a_2{\lambda_2}^n\left(\begin{array}{c}
					u_{21}\\
					u_{22}\\
					u_{23}
				\end{array}\right) + a_3{\lambda_3}^n\left(\begin{array}{c}
					u_{31}\\
					u_{32}\\
					u_{33}
				\end{array}\right)$ \\
				$x_n = a_1{\lambda_1}^nu_{13} + a_2{\lambda_2}^nu_{23} + a_3{\lambda_3}^nu_{33}$
		\end{enumerate}


































