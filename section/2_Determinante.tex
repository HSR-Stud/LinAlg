%%%%%%%%%%%%%%%%%%%%%%%%%%%%%%%%%%%%%%%%%%%%%%%%
% Determinante
%%%%%%%%%%%%%%%%%%%%%%%%%%%%%%%%%%%%%%%%%%%%%%%%
\section{Determinante}

\subsection{Definition einer Determinante}
	\begin{enumerate}
		\item $det(A)$ ändert sich nicht unter der Operation $E$
		\item Wird eine Zeile von $A$ mit $\lambda$ multipliziert, wird auch $det(A)$ mit $\lambda$ multipliziert: $det(\lambda A)=\lambda^{Anzahl Zeilen/Spalten}det(A)$
		\item $det(E) = 1$
	\end{enumerate}

\subsection{Eigenschaften der Determinante}
	\begin{itemize}
		\item Hat $A$ eine Nullzeile/Nullspalte, dann ist $det(A) = 0$
		\item Hat $A$ \underline{zwei gleiche} Zeilen/Spalten, dann ist $det(A) = 0$
		\item Ist $A$ regulär $\Leftrightarrow$ $det(A) \neq 0$ \\
			Ist $A$ singulär $\Leftrightarrow$ $det(A) = 0$
		\item Vertauscht man zwei Zeilen/Spalten, dann ändert sich das Vorzeichen der Determinante.
		\item Determinante = Fläche von Parallelogramm(2D) / Volumen von Parallelepipeds(3D)
	\end{itemize}

\subsection{Entwicklungssatz}
	\begin{enumerate}
		\item Zeile/Spalte auswählen (mit möglichst vielen Nullen)
		\item 1 Element herausnehmen
		\item Zeile und Spalte des herausgenommenen Elements abdecken
		\item Element mit der Determinante der nicht abgedeckten Elemente multiplizieren
		\item Schritt 2-4 wiederholen und zum 1. Element addieren/subtrahieren ($\rightarrow$ siehe Vorzeichen Matrix)
	\end{enumerate}
	
	Vorzeichenmatrix: $\begin{array}{|c|c|c|c|}
		\hline + & - & + & - \\
		\hline - & + & - & + \\
		\hline + & - & + & - \\
		\hline - & + & - & + \\
		\hline \end{array}$ \\ \\

	\textbf{Beispiel:} $\left|\begin{array}{ccc}
		\color{red}a & \color{green}b & \color{blue}c \\
		d & e & f \\
		g & h & i \end{array}\right| 
	= {\color{red}a} \left|\begin{array}{cc}
		e & f \\
		h & i \end{array}\right| 
	- {\color{green}b} \left|\begin{array}{cc}
		d & f \\
		g & i \end{array}\right|
	+ {\color{blue}c} \left|\begin{array}{cc}
		d & e \\
		g & h \end{array}\right|$ \\

\subsection{Wichtige Determinanten}
	$det\left(\begin{array}{cc}
		a & b \\
		c & d \end{array}\right)
	= ad - bc \qquad \qquad
	det\left(\begin{array}{ccc}
		a & b & c \\
		d & e & f \\
		g & h & i \end{array}\right)
	= \underbrace{aei + bfg + cdh - ceg - afh - bdi}_{Sarrus'sche Formel}$

\subsection{Cramsche Regel}
	$x_1= \frac{\left|\begin{array}{cccc}
		b_1 & a_{12} & \ldots & a_{1n} \\
		\vdots & \vdots & \ddots & \vdots \\
		b_n & a_{n2} & \ldots & a_{nn} \end{array}\right|}{det(A)}; \qquad
	x_2 = \frac{\left|\begin{array}{ccccc}
		a_{11} & b_1 & a_{13} & \ldots & a_{1n}\\
		\vdots & \vdots & \vdots & \ddots & \vdots \\
		a_{n1} & b_n & a_{n3} & \ldots & a_{nn} \end{array}\right|}{det(A)}$ \\ \\
	Inverse Matrix mit Cramer (Minoren): $A^{-1} = C : c_{ik} = \frac{(-1)^{k+i} \cdot det(A_{ki})}{det(A)} \longrightarrow$ 1. Index = Zeile; 2.Index = Spalte
	
	\textbf{Beispiel:} \ \ 
		$A=\left(\begin{array}{rrr} 
				-1 & -3 & 0 \\
				2 & 3 & -2 \\
				2 & 1 & -3 \\
			\end{array}\right)$ \\ \ \\
			
		$A^{-1}=\underbrace{\frac{1}{1}}_{det(A)}
			\left(\begin{array}{rrr} 
				+\underbrace{\left|\begin{array}{rr} 3 & -2 \\ 1 & -3 \\ \end{array}\right|}_{det(A_{11})} &
				-\underbrace{\left|\begin{array}{rr} -3 & 0 \\ 1 & -3 \\ \end{array}\right|}_{det(A_{21})} &
				+\underbrace{\left|\begin{array}{rr} -3 & 0 \\ 3 & -2 \\ \end{array}\right|}_{det(A_{31})} \\
			
				-\underbrace{\left|\begin{array}{rr} 2 & -2 \\ 2 & -3 \\ \end{array}\right|}_{det(A_{12})} &
				+\underbrace{\left|\begin{array}{rr} -1 & 0 \\ 2 & -3 \\ \end{array}\right|}_{det(A_{22})} &
				-\underbrace{\left|\begin{array}{rr} -1 & 0 \\ 2 & -2 \\ \end{array}\right|}_{det(A_{32})} \\
			
				+\underbrace{\left|\begin{array}{rr} 2 & 3 \\ 2 & 1 \\ \end{array}\right|}_{det(A_{13})} &
				-\underbrace{\left|\begin{array}{rr} -1 & -3 \\ 2 & 1 \\ \end{array}\right|}_{det(A_{23})} &
				+\underbrace{\left|\begin{array}{rr} -1 & -3 \\ 2 & 3 \\ \end{array}\right|}_{det(A_{33})} \\
			\end{array}\right)
			=\left(\begin{array}{rrr} 
				-7 & -9 & 6 \\
				2 & 3 & -2 \\
				-4 & -5 & 3 \\
			\end{array}\right)$
			
\subsection{Spezielle Fälle}
	$det\left(\begin{array}{cc} 
			A & 0 \\
			0 & B \\
		\end{array}\right)=det(A)det(B)$ \ \ wobei $det(A)$ und $det(B)$ Matrizen von Grösse $n*n$ sind.