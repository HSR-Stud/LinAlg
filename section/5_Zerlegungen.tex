%%%%%%%%%%%%%%%%%%%%%%%%%%%%%%%%%%%%%%%%%%%%%%%%
% Zerlegungen
%%%%%%%%%%%%%%%%%%%%%%%%%%%%%%%%%%%%%%%%%%%%%%%%

\section{Zerlegungen}

\subsection{L-U-Zerlegung}
	A = LU\\
	(A gegeben)
	\begin{enumerate}
		\item Gauss (Vorwärtsreduktion)
		\item L $\rightarrow$ Pivospalten (rot und blau), U $\rightarrow$ "Rest" (grün) 
	\end{enumerate}\ 
	$\begin{array}{|cccc|}
		\hline 
		\color{red}* & * & * & *\\
		\color{blue}* & * & * & *\\
		\color{blue}* & * & * & *\\
		\color{blue}* & * & * & *\\
		\hline
	\end{array}$
	$\rightarrow$
	$\begin{array}{|cccc|}
		\hline 
		1 & * & * & *\\
		0 & \color{red}* & * & *\\
		0 & \color{blue}* & * & *\\
		0 & \color{blue}* & * & *\\
		\hline
	\end{array}$
	$\rightarrow^{...}\rightarrow$
	$\begin{array}{|cccc|}
		\hline 
		1 & * & * & *\\
		0 & 1 & * & *\\
		0 & 0 & 1 & *\\
		0 & 0 & 0 & \color{red}*\\
		\hline
	\end{array}$
	$\rightarrow$
	$\begin{array}{|cccc|}
		\hline 
		\color{green}1 & \color{green}* & \color{green}* & \color{green}*\\
		0 & \color{green}1 & \color{green}* & \color{green}*\\
		0 & 0 & \color{green}1 & \color{green}*\\
		0 & 0 & 0 & \color{green}1\\
		\hline
	\end{array}$\ \ \ \
	$L=	\left(\begin{array}{cccc}
			\color{red}* & 0 & 0 & 0\\
		 	\color{blue}* & \color{red}* & 0 & 0\\
			\color{blue}* & \color{blue}* & \color{red}* & 0\\
			\color{blue}* & \color{blue}* & \color{blue}* & \color{red}*\\
	\end{array}\right)$\ \ \ \ \ \ 
	$U=	\left(\begin{array}{cccc}
			\color{green}* & \color{green}* & \color{green}* & \color{green}*\\
		 	0 & \color{green}* & \color{green}* & \color{green}*\\
			0 & 0 & \color{green}* & \color{green}*\\
			0 & 0 & 0 & \color{green}*\\
	\end{array}\right)$\\\\\\\\
	\textbf{Beispiel:}\ \ \ \  
	$A=	\left(\begin{array}{ccc}
		-1 &\ \ 0 &\ \ 2\\
		 \ \ 1 &\ \ 3 &\ \ 3\\
		-1 &-3 &\ \ 1\\
	\end{array}\right)$\\\\\\ 
	$\begin{array}{|ccc|}
			\hline 
			\color{red}-1 & \ \ 0 & \ \ 2\\
			\color{blue}\ \ 1 & \ \ 3 & \ \ 3\\
			\color{blue}-1 & -3 & \ \ 1\\
			\hline
	\end{array}$
	$\rightarrow$
	$\begin{array}{|ccc|}
			\hline 
			\ \ 1 & \ \ 0 & -2\\
			\ \ 0 & \color{red}\ \ 3 & \ \ 5\\
			\ \ 0 & \color{blue}-3 & -1\\
			\hline
	\end{array}$
	$\rightarrow$
	$\begin{array}{|ccc|}
			\hline 
			\ \ 1 & \ \ 0 & -2\\
			\ \ 0 & \ \ 1 & \ \ \frac 5 3\\
			\ \ 0 & \ \ 0 & \color{red}\ \ 4\\
			\hline
	\end{array}$
	$\rightarrow$
	$\begin{array}{|ccc|}
			\hline 
			\color{green}\ \ 1 & \color{green}\ \ 0 & \color{green}-2\\
			\ \ 0 & \color{green}\ \ 1 & \color{green}\ \ \frac 5 3\\
			\ \ 0 & \ \ 0 & \color{green}\ \ 1\\
			\hline
	\end{array}$\ \ \ \ \	
	$L=	\left(\begin{array}{ccc}
			\color{red}-1 &\ \ 0 &\ \ 0\\
		 	\color{blue}\ \ 1 &\color{red}\ \ 3 &\ \ 0\\
			\color{blue}-1 &\color{blue} -3 &\color{red}\ \ 4\\
	\end{array}\right)$\ \ \ \ \ \ 
	$U=	\left(\begin{array}{ccc}
			\color{green}\ \ 1 &\color{green}\ \ 0 &\color{green}-2\\
		 	\ \ 0 &\color{green}\ \ 1 &\color{green}\ \ \frac 5 3\\
			\ \ 0 &\ \ 0 &\color{green}\ \ 1\\
	\end{array}\right)$\\
		
\subsection{Q-R-Zerlegung}
	A = QR\\
	(A gegeben)
	\begin{enumerate}
		\item Q $\rightarrow$ A orthonormalisieren 
		\item R $\rightarrow$ $Q^t$A 
	\end{enumerate} 
	\textbf{Beispiel:}\ \ \ \  
	$A=	\left(\begin{array}{ccc}
		\ \ 1 &\ \ 0 &\ \ 0\\
		\ \ 1 &\ \ 1 &\ \ 0\\
		\ \ 1 &\ \ 1 &\ \ 1\\
	\end{array}\right)$
	$\rightarrow^{orthonorm.}\rightarrow$
	$\left(\begin{array}{ccc}
		\ \ \frac 1 {\sqrt 3}  & -\frac 2 {\sqrt 6} &\ \ 0\\
		\ \ \frac 1 {\sqrt 3} &\ \ \frac 1 {\sqrt 6} & -\frac 1 {\sqrt 2}\\
		\ \ \frac 1 {\sqrt 3} &\ \ \frac 1 {\sqrt 6} &\ \ \frac 1 {\sqrt 2}\\
	\end{array}\right)$ = Q\\\\\\
	R = $Q^t$A = 
	$\left(\begin{array}{ccc}
			\ \ \frac 1 {\sqrt 3}  &\ \ \frac 1 {\sqrt 3} &\ \frac 1 {\sqrt 3}\\
			 -\frac 2 {\sqrt 6} &\ \ \frac 1 {\sqrt 6} &\ \frac 1 {\sqrt 6}\\
			\ \ 0 & -\frac 1 {\sqrt 2} &\ \ \frac 1 {\sqrt 2}\\
		\end{array}\right)$ * 
	$\left(\begin{array}{ccc}
		\ \ 1 &\ \ 0 &\ \ 0\\
		\ \ 1 &\ \ 1 &\ \ 0\\
		\ \ 1 &\ \ 1 &\ \ 1\\
	\end{array}\right)$ =
	$\left(\begin{array}{ccc}
		\ \ \frac 3 {\sqrt 3} &\ \ \frac 2 {\sqrt 3} &\ \ \frac 1 {\sqrt 3}\\
		\ \ 0 &\ \ \frac 2 {\sqrt 6} &\ \ \frac 1 {\sqrt 6}\\
		\ \ 0 &\ \ 0 &\ \ \frac 1 {\sqrt 2}\\
	\end{array}\right)$

\subsection{Cholesky-Zerlegung}
	A = $LL^t$\\
	(A gegeben)\\\\ 
	$A=	\left(\begin{array}{ccc}
		\ \ l_{11} &\ \ l_{21} &\ \ l_{31}\\
		\ \ l_{12} &\ \ l_{22} &\ \ l_{32}\\
		\ \ l_{13} &\ \ l_{23} &\ \ l_{33}\\	
	\end{array}\right)$ 
	= $LL^t$ = 
	$\left(\begin{array}{ccc}
		\ \ x_1 &\ \ 0 &\ \ 0\\
		\ \ x_2 &\ \ x_4 &\ \ 0\\
		\ \ x_3 &\ \ x_5 &\ \ x_6\\	
	\end{array}\right)$ *
	$\left(\begin{array}{ccc}
		\ \ x_1 &\ \ x_2 &\ \ x_3\\
		\ \ 0 &\ \ x_4 &\ \ x_5\\
		\ \ 0 &\ \ 0 &\ \ x_6\\	
	\end{array}\right)$\\\\\\ 
	(1): $x_1*x_1 + 0*0 + 0*0 = x_1^2 = l_{11}$ $\rightarrow$ $x_1$ berechnen\\ 		
	(2): $x_2*x_1 + x_4*0 + 0*0 = x_2*x_1 = l_{12}$ $\rightarrow$ $x_2$ berechnen\\ 
	(3): $x_3*x_1 + x_5*0 + x_6*0 = x_3*x_1 = l_{13}$ $\rightarrow$ $x_3$ berechnen\\
	(4): $x_2*x_2 + x_4*x_4 + 0*0 = x_2^2 + x_4^2 = l_{22}$ $\rightarrow$ $x_4$ berechnen\\
	(5): $x_3*x_2 + x_5*x_4 + x_6*0 = x_3*x_2 + x_5*x_4 = l_{23}$ $\rightarrow$ $x_5$ berechnen\\	
	(6): $x_3*x_3 + x_5*x_5 + x_6*x_6 = x_3^2 + x_5^2 + x_6^2 = l_{33}$ $\rightarrow$ $x_6$ berechnen\\	







